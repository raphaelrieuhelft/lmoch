\documentclass[18pt]{beamer}
\usepackage{etex}
\usepackage{beamerthemesplit} % new
\usetheme[secheader]{Madrid}
%\usetheme{Warsaw}
\usecolortheme{dolphin}
\usepackage{graphicx}
%\usepackage{landscape}
\usepackage{amsmath}
\usepackage{amsfonts}
\usepackage{amssymb}
\usepackage[frenchb]{babel}
\usepackage{amstext}
\usepackage[utf8]{inputenc}
%\usepackage[T1]{fontenc}
\usepackage{pgf,tikz}
\usepackage{listings}


%\usepackage{titletoc}

\newcounter{moncompteur}
\newtheorem{q}[moncompteur]{\textbf{Question}}{}
\newtheorem{prop}[moncompteur]{\textbf{Proposition}}{}
\newtheorem{df}[moncompteur]{\textbf{Définition}}{}
\newtheorem*{df*}{\textbf{Définition}}{}
\newtheorem*{prop*}{\textbf{Proposition}}{}
\newtheorem*{theo*}{\textbf{Théorème}}{}

\newtheorem{rem}[moncompteur]{\textbf{Remarque}}{}
\newtheorem{theo}[moncompteur]{\textbf{Théorème}}{}
\newtheorem{conj}[moncompteur]{\textbf{Conjecture}}{}
\newtheorem{cor}[moncompteur]{\textbf{Corollaire}}{}
\newtheorem{lm}[moncompteur]{\textbf{Lemme}}{}
%\newtheorem{nota}[moncompteur]{\textbf{Notation}}{}
%\newtheorem{conv}[moncompteur]{\textbf{Convention}}{}
\newtheorem{exa}[moncompteur]{\textbf{Exemple}}{}
\newtheorem{ex}[moncompteur]{\textbf{Exercice}}{}
%\newtheorem{app}[moncompteur]{\textbf{Application}}{}
%\newtheorem{prog}[moncompteur]{\textbf{Algorithme}}{}
%\newtheorem{hyp}[moncompteur]{\textbf{Hypothèse}}{}
\newenvironment{dem}{\noindent\textbf{Preuve}\\}{\flushright$\blacksquare$\\}
\newcommand{\cg}{[\kern-0.15em [}
\newcommand{\cd}{]\kern-0.15em]}
\newcommand{\R}{\mathbb{R}}
\newcommand{\K}{\mathbb{K}}
\newcommand{\N}{\mathbb{N}}
\newcommand{\Z}{\mathbb{Z}}
\newcommand{\C}{\mathbb{C}}
\newcommand{\U}{\mathbb{U}}
\newcommand{\Q}{\mathbb{Q}}
\newcommand{\B}{\mathbb{B}}
\newcommand{\Prob}{\mathbb P}
\newcommand{\card}{\mathrm{card}}
\newcommand{\norm}[1]{\left\lVert#1\right\rVert}
%\pgfplotsset{compat=newest}
\newcommand{\La}{\mathcal{L}}
\newcommand{\Ne}{\mathcal{N}}
\newcommand{\D}{\mathcal{D}}
\newcommand{\Ss}{\textsc{safestay}}
\newcommand{\Sg}{\textsc{safego}}
\newcommand{\M}{\textsc{move}}
\newcommand{\E}{\mathcal{E}}
\newcommand{\V}{\mathcal V}
\newcommand{\A}{\mathcal A}
\newcommand{\T}{\mathcal T}
\newcommand{\Ca}{\mathcal C}
\setlength{\parindent}{0pt}
\newcommand{\myrightleftarrows}[1]{\mathrel{\substack{\xrightarrow{#1} \\[-.6ex] \xleftarrow{#1}}}}
\newcommand{\longrightleftarrows}{\myrightleftarrows{\rule{1cm}{0cm}}}
\newcommand{\ZnZ}{\Z/n\Z}

\lstset{frame=,
	aboveskip=3mm,
	belowskip=3mm,
	showstringspaces=false,
	columns=flexible,
	basicstyle={\small\ttfamily},
	numbers=none,
	numberstyle=\tiny\color{gray},
    morekeywords={node, returns, var, let, tel, pre, if, then, else},
	keywordstyle=\color{blue},
	commentstyle=\color{dkgreen},
	stringstyle=\color{mauve},
	breaklines=true,
	breakatwhitespace=true,
	tabsize=3
}

\graphicspath{{./}}


\setbeamertemplate{navigation symbols}{%
	%\insertslidenavigationsymbol
	%\insertframenavigationsymbol
	%\insertsubsectionnavigationsymbol
	%\insertsectionnavigationsymbol
	%\insertdocnavigationsymbol
	%\insertbackfindforwardnavigationsymbol
}
\title{Lustre Model Checker}
\author{Diane Gallois-Wong, Raphaël Rieu-Helft}
\begin{document}
\begin{frame}
	\titlepage
\end{frame}
\begin{frame}[fragile]
\frametitle{Exemple de l'énoncé}
\begin{lstlisting}
node incr (tic: bool) returns (ok: bool);
var cpt : int;
let
cpt = (0 -> pre cpt) + if tic then 1 else 0;
ok = true -> (pre cpt) <= cpt;
tel
\end{lstlisting}

$$\Delta(n) =
\begin{cases}
cpt(n) = ite(n=0, 0, cpt(n-1)) + ite(tic(n), 1, 0)
\\
ok(n) = ite(n = 0, true, cpt(n - 1) \leq cpt(n))
\end{cases}
$$
\qquad$P(n) = ok(n)$

\vspace{2em}

	$k$-induction :
	$$\Delta(0), \Delta(1), \ldots, \Delta(k) \models  P(0), P(1), \ldots,  P(k)$$
	$$\forall n>0, {\Delta(n), \ldots, \Delta(n+k), P(n), \ldots, P(n+k-1) \models P(n+k)}$$

\end{frame}

\begin{frame}
\tableofcontents
\end{frame}

\section{De l'ast typée fournie à l'ast d'AEZ}


\begin{frame}
\frametitle{De l'ast typée fournie à l'ast d'AEZ}
	
	\begin{columns}[T]
		\begin{column}{.5\linewidth}
			$
			\begin{array}{lll}
				e\;::=&
				\\
					\;\;| \; c	
				&	\text{constante}
				\\
					\;\;| \; x	
				&	\text{variable}
				\\
					\;\;| \; op(e,...,e)
				&	\text{opération}
				\\
					\;\;| \; nd(e,...,e)
				&	\text{appel de noeud}
				\\
					\;\;| \; prim(e,...,e)
				&	\text{primitive}
				\\
					\;\;| \; e\;\rightarrow\;e
				&	%\text{opération arithmétique}
				\\
					\;\;| \; pre(e)
				&	%\text{opération logique}
				\\
					\;\;| \; (e,...,e)
				&	\text{tuple}
			\end{array}
			$	
		
		\end{column}
		\begin{column}{.5\linewidth}

			
			\def\tmpcolor{gray}
			\def\newcolor{teal}
			\def\oldcolor{purple}
			
			$
			\begin{array}{lll}
				t\;::=&
				\\
					\;\;| \; c	
				&	\text{constante}
				\\
					\;\;| \; t\; op \; t 
				&	\text{opération arithmétique}
				\\
					\;\;| \; ite(f,t,t)
					%\text{it } f \text{ then } t \text{ else } t
				&	\text{opération logique}
				\\
				\only<2->{
					\textcolor{\oldcolor}
					}{\;\;| \; \varphi(t,...,t)}
				&	
				\only<2->{
					\textcolor{\oldcolor}
					}{\text{appel de fonction}}
				\\
				\onslide<2->{
					\textcolor{\newcolor}{\;\;| \; x(n-k)}
				&	\textcolor{\newcolor}{\text{variable à un instant}}
				}
				\\
				\onslide<2->{
					\textcolor{\tmpcolor}{\;\;| \; f}
				&	\textcolor{\tmpcolor}{\text{formule}}
				}
				\\
				\onslide<2->{
					\textcolor{\tmpcolor}{\;\;| \; (t,...,t)}
				&	\textcolor{\tmpcolor}{\text{tuple}}
				}
				\\
				\onslide<2->{
					\textcolor{\tmpcolor}{\;\;| \; nd(t,...,t)}
				&	\textcolor{\tmpcolor}{\text{appel de noeud}}
				}
			\end{array}
			$
			
			\vspace{1em}
									
			$
			\begin{array}{lll}
				f\;::=&
				\\
					\;\;| \; t	
				&	\text{terme}
				\\
					\;\;| \; t\; cmp \; t 
				&	\text{comparaison}
				\\
					\;\;| \; f\;op\;f
				&	\text{opération logique}
				\\
				\onslide<2->{
					\textcolor{\newcolor}{\;\;| \; n=k}
				&	\textcolor{\newcolor}{\text{temps = constante}}
				}
			\end{array}
			$
		
		\
		
		\end{column}
	\end{columns}
	
	
\end{frame}
\begin{frame}
\frametitle{De l'ast typée fournie à l'ast d'AEZ}

\begin{itemize}
\item
Étape 1 :
changement d'ast, gestion de $\rightarrow$ et $pre$.

\

\item
Étape 2 :
élimination des termes réduits à une formule.

\

\item
Étape 3 :
élimination des tuples et appels de noeuds.
\end{itemize}
\end{frame}



\subsection{Étape 1 : traduction, gestion de $\rightarrow$ et $pre$}
\begin{frame}
\frametitle{Étape 1 : traduction, gestion de $\rightarrow$ et $pre$}
	
	$n$ : terme global représentant le temps
	
	\
	
	Objectif : \qquad $x\rightarrow pre \; y \qquad\longrightarrow\qquad \text{if }n=0 \text{ then } x(n) \text{ else } y(n-1)$
	
	\
	
	\
	
	On propage $k$, décalage par rapport à $n$ :
%	$$pre(e),\;k \quad\longrightarrow\quad e,\;k+1$$
	\vspace{-1em}
	$$\Phi(pre(e),\;k) \qquad=\qquad \Phi(e,\;k+1)$$
	
	\
	
	Variables :
	\vspace{-1em}
	$$\Phi(x,\;k) \qquad=\qquad x(n-k)$$
	
	\
	
	$\rightarrow$ :
	\vspace{-1em}
	$$\Phi(e1\rightarrow e2, \;k) \qquad=\qquad \text{if }n=k \text{ then } \Phi(e1,\;k) \text{ else } \Phi(e2,\;k)$$
	
\end{frame}


\subsection{Étape 2 :
élimination des termes réduits à une formule}
\begin{frame}
\frametitle{Étape 2 :
élimination des termes réduits à une formule}
	
	$$f \qquad\longrightarrow\qquad aux$$

	où $aux$ est une variable fraîche, en ajoutant la formule
	
	$$(aux \Rightarrow f) \quad \&\& \quad (f \Rightarrow aux)$$
	
	
\end{frame}


\subsection{Étape 3 :
élimination des tuples et appels de noeuds}
\begin{frame}
\frametitle{Étape 3 :
élimination des tuples et appels de noeuds}
	
\end{frame}



\section{Implémentation du solveur}
\begin{frame}
	\frametitle{Le procédé de $k$-induction}
	Cas de base :	$$\Delta(0), \Delta(1), \ldots, \Delta(k) \models  P(0), P(1), \ldots,  P(k)$$
	Cas inductif : $$\forall n, {\Delta(n), \ldots, \Delta(n+k), P(n), \ldots, P(n+k-1) \models P(n+k)}$$

\end{frame}	
\begin{frame}
	\frametitle{Génération de code}

		
			$$\Delta(0), \Delta(1), \ldots, \color{red} \Delta(k) \color{black} \models \color{gray} P(0), P(1), \ldots, \color{red} P(k)\color{black}$$
			$$\forall n>0, {\Delta(n), \ldots, \color{red} \Delta(n+k) \color{black}, P(n), \ldots, \color{red}P(n+k-1) \color{black} \models  \color{red}P(n+k)}\color{black}$$
	\begin{itemize}
		\item{On augmente k jusqu'à échouer le cas de base ou réussir l'induction}
		\item{Code généré à partir de la syntaxe abstraite AEZ compilée}
	\end{itemize}	
\end{frame}


\def\optimisation{Une optimisation : élimination de cas particuliers}
\section{\optimisation}
\begin{frame}[fragile]
	\frametitle{\optimisation}
	
	\texttt{a -> pre (b -> pre (c -> pre d));}
	
	\
	
	est compilé en la formule :
	
	\
	
	\texttt{if n=0 then a(n) else if n=1 then b(n-1) else if n=2 then c(n-2) else d(n-3)}
	
	\bigskip
	
	Beaucoup de if... On voudrait pouvoir supposer $n>2$ pour l'induction
\end{frame}
\begin{frame}
	\frametitle{\optimisation}
	Il suffit d'ajouter des cas de base :
	
	$$\Delta(0), \Delta(1), \ldots, \color{red}\Delta(k+p)\color{black} \models P(0), P(1), \ldots, \color{red}P(k+p)\color{black}$$
	$$\color{red}\forall n>p\color{black}, {\Delta(n), \ldots, \Delta(n+k), P(n), \ldots, P(n+k-1) \models P(n+k)}$$
	
	On peut beaucoup simplifier les formules dans l'induction, voire terminer avec un $k$ plus petit.
\end{frame}
\begin{frame}[fragile]
	\frametitle{Un exemple}
	\begin{lstlisting}
	node check () returns (ok: bool);
	  var n0, n1:int; b:bool;
	  let
	    n0 = 0 -> pre n0;
	    n1 = 0 -> pre n1;
	    b = true -> pre (true -> false);
	    ok = if b then n0=n1 else true;
	  tel
	\end{lstlisting}
	
	\
	
	$1$-induction sans l'optimisation, pas besoin d'induction avec !
\end{frame}
\begin{frame}
	\frametitle{Choix de $p$}
	\begin{itemize}
	\item On prend le plus grand $i$ tel que \texttt{if(n=i)} apparaît dans le code généré
	\end{itemize}
	$\implies$ correspond à la plus grande profondeur de flèches dans le code source Lustre
	
	\begin{itemize}
		\item Pas optimal, mais on ne peut pas toujours déterminer le meilleur $p$ statiquement
	\end{itemize}
\end{frame}

\end{document}
%COMPILATION VERS AEZ
%
%Ast lustre et aez côte à côte
%
%On ajoute les constructions intermédiaires à l'ast aez
%
%Elimination :
%
%- formules
%
%- tuples et appels
%	-distribution
%	-renommage
%
%PROCEDE DE K-INDUCTION
%
%Pour chaque terme/formule, une fonction du terme n vers le terme/formule à l'instant n
%
%Formules : cas de base, induction
%
%On alterne entre ajouter un cas de base et augmenter k
%
%UNE OPTIMISATION
%
%exemple de formule pour le cas inductif avec des if n=0,1,2
%
%si on suppose n>2 => formule plus simple
%
%formules bien quantifiées avec plus de cas de base et on suppose n>2 dans l'induction
%
%comment calculer la bonne profondeur ? on prend le plus gros n=k dans le code généré => profondeur maximale de flèches
%
%Limitation : on n'élimine pas tous les cas particuliers, la condition du if peut être plus compliquée => pas praticable dans le cas général
