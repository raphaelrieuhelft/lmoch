\documentclass[18pt]{beamer}
\usepackage{etex}
\usepackage{beamerthemesplit} % new
\usetheme[secheader]{Madrid}
%\usetheme{Warsaw}
\usecolortheme{dolphin}
\usepackage{graphicx}
%\usepackage{landscape}
\usepackage{amsmath}
\usepackage{amsfonts}
\usepackage{amssymb}
\usepackage[frenchb]{babel}
\usepackage{amstext}
\usepackage[utf8]{inputenc}
%\usepackage[T1]{fontenc}
\usepackage{pgf,tikz}



%\usepackage{titletoc}

\newcounter{moncompteur}
\newtheorem{q}[moncompteur]{\textbf{Question}}{}
\newtheorem{prop}[moncompteur]{\textbf{Proposition}}{}
\newtheorem{df}[moncompteur]{\textbf{Définition}}{}
\newtheorem*{df*}{\textbf{Définition}}{}
\newtheorem*{prop*}{\textbf{Proposition}}{}
\newtheorem*{theo*}{\textbf{Théorème}}{}

\newtheorem{rem}[moncompteur]{\textbf{Remarque}}{}
\newtheorem{theo}[moncompteur]{\textbf{Théorème}}{}
\newtheorem{conj}[moncompteur]{\textbf{Conjecture}}{}
\newtheorem{cor}[moncompteur]{\textbf{Corollaire}}{}
\newtheorem{lm}[moncompteur]{\textbf{Lemme}}{}
%\newtheorem{nota}[moncompteur]{\textbf{Notation}}{}
%\newtheorem{conv}[moncompteur]{\textbf{Convention}}{}
\newtheorem{exa}[moncompteur]{\textbf{Exemple}}{}
\newtheorem{ex}[moncompteur]{\textbf{Exercice}}{}
%\newtheorem{app}[moncompteur]{\textbf{Application}}{}
%\newtheorem{prog}[moncompteur]{\textbf{Algorithme}}{}
%\newtheorem{hyp}[moncompteur]{\textbf{Hypothèse}}{}
\newenvironment{dem}{\noindent\textbf{Preuve}\\}{\flushright$\blacksquare$\\}
\newcommand{\cg}{[\kern-0.15em [}
\newcommand{\cd}{]\kern-0.15em]}
\newcommand{\R}{\mathbb{R}}
\newcommand{\K}{\mathbb{K}}
\newcommand{\N}{\mathbb{N}}
\newcommand{\Z}{\mathbb{Z}}
\newcommand{\C}{\mathbb{C}}
\newcommand{\U}{\mathbb{U}}
\newcommand{\Q}{\mathbb{Q}}
\newcommand{\B}{\mathbb{B}}
\newcommand{\Prob}{\mathbb P}
\newcommand{\card}{\mathrm{card}}
\newcommand{\norm}[1]{\left\lVert#1\right\rVert}
%\pgfplotsset{compat=newest}
\newcommand{\La}{\mathcal{L}}
\newcommand{\Ne}{\mathcal{N}}
\newcommand{\D}{\mathcal{D}}
\newcommand{\Ss}{\textsc{safestay}}
\newcommand{\Sg}{\textsc{safego}}
\newcommand{\M}{\textsc{move}}
\newcommand{\E}{\mathcal{E}}
\newcommand{\V}{\mathcal V}
\newcommand{\A}{\mathcal A}
\newcommand{\T}{\mathcal T}
\newcommand{\Ca}{\mathcal C}
\setlength{\parindent}{0pt}
\newcommand{\myrightleftarrows}[1]{\mathrel{\substack{\xrightarrow{#1} \\[-.6ex] \xleftarrow{#1}}}}
\newcommand{\longrightleftarrows}{\myrightleftarrows{\rule{1cm}{0cm}}}
\newcommand{\ZnZ}{\Z/n\Z}



\graphicspath{{./}}


\setbeamertemplate{navigation symbols}{%
	%\insertslidenavigationsymbol
	%\insertframenavigationsymbol
	%\insertsubsectionnavigationsymbol
	%\insertsectionnavigationsymbol
	%\insertdocnavigationsymbol
	%\insertbackfindforwardnavigationsymbol
}

\begin{document}
\begin{frame}
	\frametitle{Le procédé de $k$-induction}
	
	$$\Delta(0), \Delta(1), \ldots, \Delta(k) \models P(0), P(1), \ldots, P(n)$$
	$$\forall n>0, {\Delta(n), \ldots, \Delta(n+k), P(n), \ldots, P(n+k-1) \models P(n+k)}$$
\end{frame}	
\begin{frame}
	\frametitle{Implémentation}
		$$\Delta(0), \Delta(1), \ldots, \Delta(k) \models P(0), P(1), \ldots, P(n)$$
		$$\forall n>0, {\Delta(n), \ldots, \Delta(n+k), P(n), \ldots, P(n+k-1) \models P(n+k)}$$
	\begin{itemize}
		\item{On augmente k jusqu'à échouer le cas de base ou réussir l'induction}
		\item{Code généré à partir de la syntaxe abstraite AEZ compilée}
	\end{itemize}	
\end{frame}
\begin{frame}[fragile]
	\frametitle{Une optimisation}
	
	\texttt{n1 = 0 -> pre (0 -> pre (0 -> pre n1));}
	
	\
	
	est compilé en la formule :
	
	\
	
	\texttt{n1(n)  =  (if n=0 then 0 else (if n=1 then 0 else (if n=2 then 0 else n1(n-3))))}
	
	\bigskip
	
	Beaucoup de if... On voudrait pouvoir supposer $n>2$ pour l'induction
\end{frame}
\begin{frame}
	
\end{frame}
\end{document}
%COMPILATION VERS AEZ
%
%Ast lustre et aez côte à côte
%
%On ajoute les constructions intermédiaires à l'ast aez
%
%Elimination :
%
%- formules
%
%- tuples et appels
%	-distribution
%	-renommage
%
%PROCEDE DE K-INDUCTION
%
%Pour chaque terme/formule, une fonction du terme n vers le terme/formule à l'instant n
%
%Formules : cas de base, induction
%
%On alterne entre ajouter un cas de base et augmenter k
%
%UNE OPTIMISATION
%
%exemple de formule pour le cas inductif avec des if n=0,1,2
%
%si on suppose n>2 => formule plus simple
%
%formules bien quantifiées avec plus de cas de base et on suppose n>2 dans l'induction
%
%comment calculer la bonne profondeur ? on prend le plus gros n=k dans le code généré => profondeur maximale de flèches
%
%Limitation : on n'élimine pas tous les cas particuliers, la condition du if peut être plus compliquée => pas praticable dans le cas général
